\documentclass[12pt]{article}
\usepackage{amsmath}
\usepackage[left=1in, right=1in]{geometry}
\usepackage{parskip}



\begin{document}

\begin{center}
\textbf{\huge{Evolution of Neural Networks in Flatworld}}\\
\vspace{5mm}
\Large{Taylor Berger \& James Vickers}\\
\vspace{3mm}
\Large{\emph{Department of Computer Science}}\\
\Large{\emph{University of New Mexico}}\\
\vspace{3mm}
\normalsize{tberge01@cs.unm.edu \& jvick3@unm.edu}\\
\end{center}

\section*{Abstract}
\paragraph{}\emph{Flatworld is a two dimensional environmet explored and
  interacted with by a set of programmable agents. The world is
  scattered with edible food that may be beneficial, neutral or
  poisonous in nature. Without the direct ability to identify which
  food is good or bad, an agent must learn to separate the food into
  their respective categories based on the sensory information
  available to the agent. In this paper, we show how to evolve a smart
  agent over a series of neural network architectures with the
  objective of living as long as possible. We designed a set of
  architectures that evolve certain capabilities over time and show
  the effectiveness of those evolutions as functions operating on the
  expected lifetime of the agent}
\linespread{1.5}
\setlength{\parindent}{1.5cm}

\section{Introduction}
\paragraph{} This project is using the predefined code base called 
Flatworld (~\ref{tcaudell}). The initial setup gives little information
about the world and objects inside so the objective of our project is
to evolve a neural network controlling our agent to survive as long as
possible. 
\section{Description of Anticipated Designs}
\section{Learning Approach}
\section{Evaluation Approach}
\section{Analysis/Presentation Approach}
\section{Preliminary Results}
\section{Known Issues}
\section{Referenced}

\end{document}
